\documentclass{article}
\usepackage{float}
%\usepackage{hyperref}
\usepackage{fancyhdr}
\usepackage{graphicx}
\usepackage{fancyvrb}
%\usepackage[T1]{fontenc}
\fancypagestyle{first}{
  \lhead{Experiment 3
  
  Date 06-03-2023}
  \rhead{}}
  \fancypagestyle{third}{
  \lhead{}
  \rhead{SYSTEM CALLS}}
\setcounter{page}{35}
% Report TITLE
\title{\textbf {SYSTEM CALLS}}
\date{\vspace{-5ex}}
\begin{document}
\maketitle
\thispagestyle{first}
\pagestyle{third}
%\pagestyle{first}
\section*{\Large Aim}
To familiarize the System calls such as fork, exec, getpid, exit, wait,
close, stat, opendir, readdir
\section{\Large fork}
\subsection{Aim}
\begin{Verbatim}[tabsize = 4]
To familiarize the system call fork
\end{Verbatim}
\subsection{Description}
\begin{Verbatim}[tabsize = 4]
1. Start
2. Use fork to create a child process
3. Print 'hello world"
4. Stop
\end{Verbatim}

\subsection{Program}
\begin{Verbatim}[tabsize = 4]
#include <stdio.h>
#include <sys/types.h>
#include <unistd.h>
int main()
{
	fork();
	printf("Hello world!\n");
	return 0;
}

\end{Verbatim}

\subsection{Sample Input and Output}
\begin{figure}[H]
    \centering
    \includegraphics[width = 10cm ]{1.png}
    \caption{Output}
    \label{fig:my_label2}
\end{figure}
\section{\Large exec}
\subsection{Aim}
To familiarize with the system call exec
\subsection{Description}
\begin{Verbatim}[tabsize = 4]
System call is used to replace the process’s memory space with a new program.
The exec system call is used to execute a file which is residing in an active process.
\end{Verbatim}
%\rhead{2. REVERSING ARGUMENTS}
\subsection{Program}
\begin{Verbatim}[tabsize = 4]
#include <stdio.h>
#include <unistd.h>
#include <stdlib.h>

int main()
{
	char *args[]={"./execdemo",NULL};
	execvp(args[0],args);
	printf("End\n");
}

\end{Verbatim}
\subsection{Sample input and output}
\begin{figure}[H]
    \centering
    \includegraphics[width = 10cm ]{2.png}
    \caption{Output}
    \label{fig:my_label2}
\end{figure}
\section{\Large getpid()}
\subsection{Aim}
\begin{Verbatim}[tabsize = 4]
To familiarize with the system call getpid()
\end{Verbatim}
\newpage
\subsection{Algorithm}
%\rhead{3. GROSS SALARY CALCULATION}
\begin{Verbatim}[tabsize = 4]
1. Start
2. Get pid=fork()
3. If pid==0, then
    print process id: getpid()
4. Stop
\end{Verbatim}
\subsection{Program}
\begin{Verbatim}[tabsize = 4]
#include <stdio.h>
#include <unistd.h>
int main()
{
	int pid;
	pid = fork();
	if (pid == 0)
	{
		printf("Process id : %d \n",getpid());
	}
	return 0;
}

\end{Verbatim}
\thispagestyle{third}
\subsection{Sample Input and Output}
\begin{figure}[H]
    \centering
    \includegraphics[width = 10cm ]{3.png}
    \caption{Output}
    \label{fig:my_label2}
\end{figure}
\section{\Large exit() }
\subsection{Aim}
To familiarize with the system call exit()
\subsection{Algorithm}
\begin{Verbatim}[tabsize = 4]
1. Start
2. Fork the process and print start
3. Use exit(0) to exit from the child process and print exiting
4. print end
5. return(0)
6. Stop
\end{Verbatim}
%\rhead{4. SMALLEST OF 3 NOS}
\thispagestyle{third}
\subsection{Program}
\begin{Verbatim}[tabsize = 4]
#include <stdio.h>
#include <stdlib.h>
#include<unistd.h>
#include <sys/types.h>

int main () {
   printf("Start of the program....\n");
   fork();
   
   printf("Exiting the program....\n");
   exit(0);

   printf("End of the program....\n");

   return(0);
}

\end{Verbatim}
\subsection{Sample input and output}
\begin{figure}[H]
    \centering
    \includegraphics[width = 10cm ]{4.png}
    \caption{Output}
    \label{fig:my_label2}
\end{figure}
\section{\Large wait}
\subsection{Aim}
To familiarize with the system call wait()
\subsection{Algorithm}
\begin{Verbatim}[tabsize = 4]
1. Start
2. pid=fork()
3. if pid<0 
        print fork failed and exit
    else if pid ==0
        execute child process
    else
        use wait() for parent process to wait for child to terminate
        execute parent process
4. Stop

\end{Verbatim}
%\rhead{5. REVERSING NUMBER}
\subsection{Program}
\begin{Verbatim}[tabsize = 4]
#include<stdio.h>
#include<unistd.h>
#include<stdlib.h>
#include<sys/wait.h>
int main()
{
    int pid=fork(); //creating child
    if(pid<0)
    {
        printf("fork failed!\n");
    }
    else if(pid==0)
    {
        printf("\nchild process %d executing\n",getpid());
        printf("calling program prg.c from child using exec\n");
        char* argv[]={"prg",NULL};
        execv("./prg",argv);
    }
    else
    {
        printf("parent process %d executing\n",getpid());
        int status;
        printf("parent waiting for child to terminate\n");
        int pid=wait(&status);
        printf("child process %d exited-parent wait over\n",pid);
        printf("parent process %d exits\n",getpid());
    }
}


\end{Verbatim}
\thispagestyle{third}
\subsection{Sample input and output}
\begin{figure}[H]
    \centering
    \includegraphics[width = 10cm ]{5.png}
    \caption{Output}
    \label{fig:my_label2}
\end{figure}

\section{\Large close()}
\subsection{Aim}
To familiarize with the system call close()
\subsection{Algorithm}
\begin{Verbatim}[tabsize = 4]
1. Start
2. Open file testfile.txt using filedesc descriptor
3. if filedesc<0
        print outside close
        return 1
    else if close(filedesc)<0
        print inside close and return 0
4. Stop
\end{Verbatim}
%\rhead{6. ARMSTRONG NUMBERS}
\subsection{Program}
\begin{Verbatim}[tabsize = 4]
#include <stdlib.h>
#include <fcntl.h>
#include<unistd.h>
#include<stdio.h>
 
int main()
{
    size_t filedesc = open("testfile.txt", O_WRONLY | O_CREAT);
    if(filedesc < 0)
    {
    	printf("Outside close\n");
        return 1;
    }
 
    if(close(filedesc) < 0)
    {
    	printf("Inside close\n");
        return 1;
    }
    else
    {
    	printf("File present\n");
    }
 
    return 0;
}

\end{Verbatim}
\thispagestyle{third}
\subsection{Sample input and output}
\begin{figure}[H]
    \centering
    \includegraphics[width = 10cm ]{6.png}
    \caption{Output}
    \label{fig:my_label2}
\end{figure}
\section{\Large Stat}
\subsection{Aim}
To familiarize with the system call stat()
%\rhead{7. NUMBER PATTERN}
\subsection{Algorithm}
\begin{Verbatim}[tabsize = 4]
1. Start
2. Use stat() to access various information about a file
3. Stop
\end{Verbatim}
\subsection{Program}
\begin{Verbatim}[tabsize = 4]
#include <sys/types.h>
#include <sys/stat.h>
#include <stdio.h>
#include <time.h>

int main() {
  struct stat info;

  if (stat("/", &info) != 0)
    perror("stat() error");
  else {
    puts("stat() returned the following information about root f/s:");
    printf("  inode:   %d\n",   (int) info.st_ino);
    printf(" dev id:   %d\n",   (int) info.st_dev);
    printf("   mode:   %08x\n",       info.st_mode);
    printf("  links:   %ld\n",         info.st_nlink);
    printf("    uid:   %d\n",   (int) info.st_uid);
    printf("    gid:   %d\n",   (int) info.st_gid);
  }
  return 0;
}
\end{Verbatim}
\subsection{Sample input and output}
\thispagestyle{third}
\begin{figure}[H]
    \centering
    \includegraphics[width = 10cm ]{7.png}
    \caption{Output}
    \label{fig:my_label2}
\end{figure}
\section{\Large opendir() and closedir() }
\subsection{Aim}
To familiarize with the system call opendir() and closedir()
\subsection{Algorithm}
\begin{Verbatim}[tabsize = 4]
1. Start
2. fl=opendir(dirname) to get directory stream pointer
3. e=readdir(fl) to get pointer to the directory structure and print 
various details in the directory
4. close(fl) to close the directory stream
5. Stop
\end{Verbatim}
%\rhead{8. PASSWORD VALIDATION}
\subsection{Program}
\begin{Verbatim}[tabsize = 4]
#include <stdio.h>
#include <dirent.h>

int main()
{
    DIR *folder;
    struct dirent *entry;
    int files = 0;

    folder = opendir("samp");
    if(folder == NULL)
    {
        perror("Unable to read directory");
        return(1);
    }

    while( (entry=readdir(folder)) )
    {
        files++;
        printf("File %3d: %s\n",
                files,
                entry->d_name
              );
    }

    closedir(folder);

    return(0);
}
\end{Verbatim}
\thispagestyle{third}
\subsection{Sample input and output}
\begin{figure}[H]
    \centering
    \includegraphics[width = 10cm ]{8.png}
    \includegraphics[width = 10cm ]{8.1.png}
    \caption{Output}
    \label{fig:my_label2}
\end{figure}

\section*{\Large Result}
    Basic shell programs were written and executed successfully.

\end{document}