\documentclass{article}
\usepackage{float}
%\usepackage{hyperref}
\usepackage{fancyhdr}
\usepackage{graphicx}
\usepackage{fancyvrb}
%\usepackage[T1]{fontenc}
\fancypagestyle{first}{
  \lhead{Experiment 4
  
  Date 06-03-2023}
  \rhead{}}
  \fancypagestyle{third}{
  \lhead{}
  \rhead{I/O SYSTEM CALLS}}
\setcounter{page}{35}
% Report TITLE
\title{\textbf {FAMILIARIZATION OF I/O SYSTEM CALLS}}
\date{\vspace{-5ex}}
\begin{document}
\maketitle
\thispagestyle{first}
\pagestyle{third}
%\pagestyle{first}
\section*{\Large Aim}
To familiarize the basic I/O ystem calls such as open,close, read and write
%\section{\Large fork}
\section{open()}
%\begin{Verbatim}[tabsize = 4]

%\end{Verbatim}
\subsection{Description}
\begin{Verbatim}[tabsize = 4]
A call to it creates a new open file descriptor, and is used to open a file. 
The open() function returns a new file descriptor. Unsuccessful attempt returns -1.
\end{Verbatim}
\section{read()}
\subsection{Description}
\begin{Verbatim}[tabsize = 4]
It Used to retrieve data from a file stored in a file system. 
The value returned is the number of bytes read (or -1 for error) 
and the file position is moved forward by this number.

\end{Verbatim}
\section{write()}
\subsection{Description}
\begin{Verbatim}[tabsize = 4]
It writes data from a buffer to a file/device. The value returned is 
the number of bytes written (or -1 for error)successfully.

\end{Verbatim}
\section{close()}
\subsection{Description}
\begin{Verbatim}[tabsize = 4]
Used to close the file which pointed by file descriptor. Return 0 on success 
and -1 on error. 

\end{Verbatim}

\subsection{Program}
\begin{Verbatim}[tabsize = 4]
#include<stdio.h>
#include<string.h>
#include<unistd.h>
#include<fcntl.h>

int main (void)
{
	int fd[2];
	char buf1[12] = "hello world\n";
	char buf2[12];
	
	fd[0] = open("foobar.txt", O_RDWR);		
	fd[1] = open("foobar.txt", O_RDWR);
	
	write(fd[0], buf1, strlen(buf1));		
	write(1, buf2, read(fd[1], buf2, 12));

	close(fd[0]);
	close(fd[1]);

	return 0;
}
\end{Verbatim}

\section*{Sample Input and Output}
\begin{figure}[H]
    \centering
    \includegraphics[width = 10cm ]{1.png}
    \caption{Output}
    \label{fig:my_label2}
\end{figure}

\section*{\Large Result}
    Basic I/O system calls were familiarized and executed successfully.

\end{document}