\documentclass{article}
\usepackage{float}
%\usepackage{hyperref}
\usepackage{fancyhdr}
\usepackage{graphicx}
\usepackage{fancyvrb}
%\usepackage[T1]{fontenc}
\fancypagestyle{first}{
  \lhead{Experiment 2
  
  Date 02-03-2023}
  \rhead{}}
  \fancypagestyle{third}{
  \lhead{Sample Input and Output}}
\setcounter{page}{35}
% Report TITLE
\title{\textbf {SHELL PROGRAMMING}}
\date{\vspace{-5ex}}
\begin{document}
\maketitle
\thispagestyle{first}
%\pagestyle{first}
\section*{\Large Aim}
To write and execute basic shell programs.
\section{\Large Files to uppercase}
\subsection{Aim}
\begin{Verbatim}[tabsize = 4]
Accept one or more file name as arguments and convert the file contents to
uppercase, provided they exist in the current directory.
\end{Verbatim}
\subsection{Algorithm}
\begin{Verbatim}[tabsize = 4]
1. Start
2. Read file name from command line
3. For each file convert the contents to uppercase.
4. Stop
\end{Verbatim}
\subsection{Program}
\begin{Verbatim}[tabsize = 4]
for name in $@
do
echo "file name:"$name
tr [:lower:] [:upper:] < $name
echo " "
done
\end{Verbatim}
\pagestyle{third}
\subsection{Sample Input and Output}
\begin{figure}[H]
    \centering
    \includegraphics[width = 10cm ]{1.png}
    \caption{Output}
    \label{fig:my_label2}
\end{figure}
\section{\Large Reversing Arguments}
\subsection{Aim}
Accept any number of arguments and prints them in the reverse order.
\subsection{Algorithm}
\begin{Verbatim}[tabsize = 4]
1. Start
2. Read the arguments and store in an array
3. Print the array in reverse order.
4. Stop
\end{Verbatim}
\rhead{2. REVERSING ARGUMENTS}
\subsection{Program}
\begin{Verbatim}[tabsize = 4]
a=("$@")
n=$#
for(( i=$n-1; i>=0; i-- ))
do
echo "${a[$i]}"
done
\end{Verbatim}
\subsection{Sample input and output}
\begin{figure}[H]
    \centering
    \includegraphics[width = 10cm ]{2.png}
    \caption{Output}
    \label{fig:my_label2}
\end{figure}
\section{\Large Gross Salary Calculation}
\subsection{Aim}
\begin{Verbatim}[tabsize = 4]
Compute the gross salary of an employee according to the following rules
(i) if basic salary is less than 1500 then HRA =10% of the basic and DA =90% of the 
basic. 
(ii) If basic salary is greater than or equal to 1500 then HRA =Rs.500 and DA=98% 
of the basic
\end{Verbatim}
\newpage
\subsection{Algorithm}
\rhead{3. GROSS SALARY CALCULATION}
\begin{Verbatim}[tabsize = 4]
1. Start
2. Read basicSalary
3. If basicSalary < 1500, then HRA = (.1) * basicSalary and DA = (.9) * basicSalary
4. Else HRA = 500 and DA = (98/100) * basicSalary
5. Print HRA and DA
6. Stop
\end{Verbatim}
\subsection{Program}
\begin{Verbatim}[tabsize = 4]
read -p "Basic pay:" basic
if [ $basic -lt 1500 ]
then
  hra=$(echo "$basic * 0.1" | bc)
  da=$(echo "$basic * 0.9" | bc)
else
  hra=500
  da=$(echo "$basic * 0.98" | bc)
fi
gross=$(echo "$basic + $hra + $da" | bc)
echo "Gross Salary: Rs. $gross"
\end{Verbatim}
\thispagestyle{third}
\subsection{Sample Input and Output}
\begin{figure}[H]
    \centering
    \includegraphics[width = 10cm ]{3.png}
    \caption{Output}
    \label{fig:my_label2}
\end{figure}
\section{\Large Smallest of three numbers}
\subsection{Aim}
Find smallest of 3 numbers that are read from keyboard.
\subsection{Algorithm}
\begin{Verbatim}[tabsize = 4]
1. Start
2. Read a, b, c
3. If a < b and a < c, then print a
4. Else if b < a and b < c, then print b
5. Else print c
6. Stop
\end{Verbatim}
\rhead{4. SMALLEST OF 3 NOS}
\thispagestyle{third}
\subsection{Program}
\begin{Verbatim}[tabsize = 4]
read -p "Enter 1st number " num1
read -p "Enter 2nd number " num2
read -p "Enter 3rd number " num3
if [ $num1 -lt $num2 ] && [ $num1 -lt $num3 ]
then
  smallest=$num1
elif [ $num2 -lt $num1 ] && [ $num2 -lt $num3 ]
then
  smallest=$num2
else
  smallest=$num3
fi
echo "The smallest number is: $smallest"

\end{Verbatim}
\subsection{Sample input and output}
\begin{figure}[H]
    \centering
    \includegraphics[width = 10cm ]{4.png}
    \caption{Output}
    \label{fig:my_label2}
\end{figure}
\section{\Large Reversing number}
\subsection{Aim}
Print a number in reverse order.
\subsection{Algorithm}
\begin{Verbatim}[tabsize = 4]
1. Start
2. Enter the number n
3. while n>0,do step 4 and 5
4. r = r*10+ n%10
5. n = n/10
6. print r
7. Stop
\end{Verbatim}
\rhead{5. REVERSING NUMBER}
\subsection{Program}
\begin{Verbatim}[tabsize = 4]
read -p "Enter number" n
r=0
while [ $n -gt 0 ] ; do
r=$((r*10+n%10))
n=$((n/10))
done
echo $r
\end{Verbatim}
\thispagestyle{third}
\subsection{Sample input and output}
\begin{figure}[H]
    \centering
    \includegraphics[width = 10cm ]{5.png}
    \caption{Output}
    \label{fig:my_label2}
\end{figure}

\section{\Large Armstrong Numbers}
\subsection{Aim}
Print all armstrong numbers between two numbers.
\subsection{Algorithm}
\begin{Verbatim}[tabsize = 4]
1. Start
2. Input n1 and n2
3. i = n1
4. While i <= n2, repeat steps 5 to 12
5. a = i
6. s = 0
7. l=length(i)
8. While a > 0, repeat steps 8 to 10
9. k = a % 10
10. s = s + k^l
11. a = a / 10
12. If s = i, print i
13. Increment i
14. Stop
\end{Verbatim}
\rhead{6. ARMSTRONG NUMBERS}
\subsection{Program}
\begin{Verbatim}[tabsize = 4]
read -p "Enter number 1 :" n1
read -p "Enter number 2 :" n2
for (( i=$n1 ; i<=$n2 ; i++ ))
do
    s=0
    a=$i
    l=${#a}
    while [ $a -gt 0 ] 
    do
        k=$((a%10))
        s=$((k**l+s))
        a=$((a/10))
    done
    if [ $s -eq $i ]
    then
    echo $i
    fi
done
\end{Verbatim}
\thispagestyle{third}
\subsection{Sample input and output}
\begin{figure}[H]
    \centering
    \includegraphics[width = 10cm ]{6.png}
    \caption{Output}
    \label{fig:my_label2}
\end{figure}
\section{\Large Number Pattern}
\subsection{Aim}
Print the following pattern upto n rows, for a given n.\\
1\\
2 2\\
3 3 3\\
4 4 4 4\\
.\\
.\\
n n n n n ..\\
\rhead{7. NUMBER PATTERN}
\subsection{Algorithm}
\begin{Verbatim}[tabsize = 4]
1. Start
2. Read n
3. i = 1
4. While i <= n, repeat steps 5 to 10
5. j = 0
6. While j <i, repeat steps 7 and 8
7. Print i
8. Increment j
9. Print newline
10. Increment i
11. Stop
\end{Verbatim}
\subsection{Program}
\begin{Verbatim}[tabsize = 4]
read -p "Enter n" n
for (( i=1;i<=n;i++ ))
do
    for(( j=0;j<i;j++ ))
    do
        echo -n $i
    done
    echo ""
done
\end{Verbatim}
\subsection{Sample input and output}
\thispagestyle{third}
\begin{figure}[H]
    \centering
    \includegraphics[width = 10cm ]{7.png}
    \caption{Output}
    \label{fig:my_label2}
\end{figure}
\section{\Large Password Validation}
\subsection{Aim}
Validate password strength. Here are a few assumptions for the password string.Length should
be minimum of 8 characters.Should contain both small and capital case letters, atleast a digit
and an underscore.If the password doesn’t comply with any of the above conditions, then the
script should report it as a Weak Password.
\subsection{Algorithm}
\begin{Verbatim}[tabsize = 4]
1. Start
2. Read password
3. If length of password is greater than 8 and password contains atleast 1
uppercase,lowercase,digit and underscore then print "Strong password"
4. Else print "Weak password"
5. Stop
\end{Verbatim}
\rhead{8. PASSWORD VALIDATION}
\subsection{Program}
\begin{Verbatim}[tabsize = 4]
read -p "Enter your password: " password
if [ ${#password} -ge 8 ] && [[ $password =~ [a-z] ]] &&
[[ $password =~ [A-Z] ]] && [[ $password =~ [0-9] ]] && [[ $password =~ [_] ]];
then
    echo "Strong Password"
else
    echo "Weak Password"
fi
\end{Verbatim}
\thispagestyle{third}
\subsection{Sample input and output}
\begin{figure}[H]
    \centering
    \includegraphics[width = 10cm ]{8.png}
    \includegraphics[width = 10cm ]{8.1.png}
    \caption{Output}
    \label{fig:my_label2}
\end{figure}
\section{\Large Binary and Hexadecimal}
\rhead{9. BINARY AND HEXADECIMAL}
\subsection{Aim}
Print the binary and hexadecimal equivalent of a decimal number.
\subsection{Algorithm}
\begin{Verbatim}[tabsize = 4]
1. Start
2. Read Decimal number
3. Initialize b=d,i=o,a[]
4. while(d>0) do steps 5 to 7
5. a[i]=d%2
6. d=d/2
7. i=i+1
8. print a from index i-1 to 0
9. set i=0d=b,a[]
10. while(d>0) do steps 5 to 7
11. r=d%16
12. if (r<10) then a[i]=r
13. else 
    13.1 if r=10 a[i]="A"
    13.2 if r=11 a[i]="B"
    13.3 if r=12 a[i]="C"
    13.4 if r=13 a[i]="D"
    13.5 if r=14 a[i]="E"
    13.6 if r=15 a[i]="F"
14. d=d/16
15. i=i+1
16. print a from index i-1 to 0
17. Stop
\end{Verbatim}
%\rhead{8. PASSWORD VALIDATION}
\lhead{Program}
\subsection{Program}
\begin{Verbatim}[tabsize = 4]
read -p "Enter decimal number" d
a=()
b=$d
i=0
while [ $d -gt 0 ]
do
    a[i]=$((d%2))
    d=$((d/2))
    i=$((i+1))
done
echo -n "Binary of $b is "
for (( j= $i-1;j>=0;j-- ))
do
    echo -n "${a[$j]}"
done
echo ""
a=()
i=0
d=$b
while [ $d -gt 0 ]
do
    r=$((d%16))
    if [ $r -lt 10 ]
    then
        a[i]=$r
    else
        case $r in
            10) a[i]="A";;
            11) a[i]="B";;
            12) a[i]="C";;
            13) a[i]="D";;
            14) a[i]="E";;
            15) a[i]="F";;
        esac
    fi
    d=$((d/16))
    i=$((i+1))
done
echo -n "Hexadecimal of $b is "
for (( j= $i-1;j>=0;j-- ))
do
    echo -n "${a[$j]}"
done
echo ""
\end{Verbatim}
\thispagestyle{third}
\subsection{Sample input and output}
\begin{figure}[H]
    \centering
    \includegraphics[width = 10cm ]{9.png}
    \caption{Output}
    \label{fig:my_label2}
\end{figure}
\section{\Large 3 digit numbers from 0,1,2,3}
\subsection{Aim}
Generate all 3 digit numbers that contain only the digits 0, 1, 2, 3.(number does not start with 0)
\subsection{Algorithm}
\begin{Verbatim}[tabsize = 4]
1. Start
2. for i=1 to i=3 do steps 3 and 4
3. for j=0 to j=3 do step 4 
4. for k=0 to k=3 print (ijk)
5. Stop
\end{Verbatim}
\rhead{10. 3 DIGIT NUMBERS}
\subsection{Program}
\begin{Verbatim}[tabsize = 4]
for (( i=1; i<=3; i++ ))
do
    for (( j=0; j<=3; j++ ))
    do
        for(( k=0; k<=3; k++ ))
        do
            echo $i$j$k
            c=$((c+1))
        done
    done
done
\end{Verbatim}
\thispagestyle{third}
\subsection{Sample input and output}
\begin{figure}[H]
    \centering
    \includegraphics[width = 10cm ]{}
    \caption{Output}
    \label{fig:my_label2}
\end{figure}
\section{\Large Summarize directory}
\rhead{11. SUMMARIZE DIRECTORY}
\subsection{Aim}
Create the command ‘summarize directory’ which takes a directory as input and summarizes
the contents in the directory. This command has to print the a. Total Number of Files in the
directory b. List of all extensions of files present c. Count of files having each extension in new
lines The command works recursively for all the subdirectories.
\subsection{Algorithm}
\begin{Verbatim}[tabsize = 4]
1. Start
2. Read path from terminal
3. if directory donot exit print directory not found and stop
4. if number of arguments is 0,then print Usage:Summarize_directory and stop
5. Find the number of files and print it.
6. Iterate through all files and store its extensions and print it.
7. Extract unique extensions and find count of each and print  it.
8. for all subdirectories repeat steps 5 to 7.
9.Stop
\end{Verbatim}
\subsection{Program}
\begin{Verbatim}[tabsize = 4]
summarize_dir() {
    dir="$1"
    file_ext=""
    count=0
    num_files=$(find "$dir" -type f | wc -l)
    echo "Total Number of Files in the Directory: $num_files"
    for file in $(find "$dir" -type f); do
        file_ext=$(echo "${file##*.}")
        echo "$file_ext"
    done | sort -u > extensions.txt
    while read ext; do
        count=$(find "$dir" -type f -name "*.$ext" | wc -l)
        echo "$ext: $count"
    done < extensions.txt
    rm extensions.txt
}
if [ -z "$1" ]; then
    echo "Usage: summarize_directory [directory]"
    exit 1
fi
if [ ! -d "$1" ]; then
    echo "Error: Directory does not exist."
    exit 1
fi
for subdir in $(find "$1" -type d); do
    summarize_dir "$subdir"
done
\end{Verbatim}
\thispagestyle{third}
\subsection{Sample input and output}
\begin{figure}[H]
    \centering
    \includegraphics[width = 10cm ]{11.png}
    \caption{Output}
    \label{fig:my_label2}
\end{figure}
\section{\Large Pass/Fail}
\rhead{12. PASS/FAIL}
\subsection{Aim}
Given a file containing the marks obtained by students for 3 subjects in an exam. In order to
pass, student should score at least 50 marks in every subject. The file has one record(line) for
each student in the following format: rollnumber subject1 subject2 subject3 Write a script to
print pass/ fail status of each student in the following format: rollnumber pass/fail.
\subsection{Algorithm}
\begin{Verbatim}[tabsize = 4]
1. Start
2. Read File name
3. For each line in the file do step steps 4 and 5
4. if all 3 marks are greater than or equal to 50 ,set status="Pass" else set status ="fail".
5. print(rollno ,status)
6. Stop
\end{Verbatim}
\subsection{Program}
\begin{Verbatim}[tabsize = 4]
read -p "Enter the file name: " file
while read line 
do
    echo "$line"
    fields=($line)
    roll=${fields[0]}
    sub1=${fields[1]}
    sub2=${fields[2]}
    sub3=${fields[3]}
    if (( sub1 >= 50 && sub2 >= 50 && sub3 >= 50 ))
    then
        status="pass"
    else
        status="fail"
    fi
    echo "$roll $status"
done < $file

\end{Verbatim}
\thispagestyle{third}
\subsection{Sample input and output}
\begin{figure}[H]
    \centering
    \includegraphics[width = 10cm ]{13.png}
    \caption{Output}
    \label{fig:my_label2}
\end{figure}
\section{\Large Palindromic Prime}
\rhead{13. PALINDROMIC PRIME}
\subsection{Aim}
Find the smallest prime number greater than n which is palindromic.
\subsection{Algorithm}
\begin{Verbatim}[tabsize = 4]
1. Start
2. Read n
3. Initialize f=0
4. while(f==0) do steps
5. n=n+1
6. for i=2 to squareroot(n)
    6.1 if n%i =0 set f=1 and break
7. if f=0 then
    7.1 set r=0,num=n
    7.2 while(num >0) do
            r=r*10+num%10
            num=num/10
    7.3 if r=n then
            set f=1
            print n
    7.4 else
            set f=0
8. else set f=0
9. Stop
\end{Verbatim}
\subsection{Program}
\begin{Verbatim}[tabsize = 4]
read -p "Enter n" n
f=0
while [ $f -eq 0 ]
do
    n=$((n+1))
    for (( i=2; i*i<=n; i++ ))
    do 
        if [ $((n%i)) -eq 0 ]
        then
            f=1
            break
        fi
    done
    if [ $f -eq 0 ]
    then
        r=0
        num=$n
        while [ $num -gt 0 ] 
        do
        r=$((r*10+num%10))
        num=$((num/10))
        done
        if [ $r -eq $n ]
        then
            f=1
            echo "The next palindromic prime number is $n"
        else
            f=0
        fi
    else
        f=0
    fi
done

\end{Verbatim}

\thispagestyle{third}
\subsection{Sample input and output}
\begin{figure}[H]
    \centering
    \includegraphics[width = 10cm ]{pal.png}
    \caption{Output}
    \label{fig:my_label2}
\end{figure}
\section{\Large Pattern}
\subsection{Aim}
\begin{Verbatim}[tabsize = 4]
Shell program to print the following pattern upto n rows, for a given n.
    *
  * * *
* * * * *
  * * *
    *
\end{Verbatim}
\subsection{Algorithm}
\rhead{14. PATTERN}
\begin{Verbatim}[tabsize = 4]
1. Start
2. Read n
3. if n is even set r=n/2 else set r=n/2+1
4. Initialize c=0
5. for i=1 to n do steps 6 to 12
6. Set j=1
7. while(j<=r-c-1) do
    7.1 print " "
    7.2 j=j+1
8. while(j<=r+c) do
    7.1 print "*"
    7.2 j=j+1
9. if (i<r) then Increment c
10. if (i=r and ni odd) then Decrement c
11. Else Decrement c
12. Print newline
13. Stop
\end{Verbatim}
\subsection{Program}
\begin{Verbatim}[tabsize = 4]
read -p "Enter n" n
f=0
while [ $f -eq 0 ]
do
    n=$((n+1))
    for (( i=2; i*i<=n; i++ ))
    do 
        if [ $((n%i)) -eq 0 ]
        then
            f=1
            break
        fi
    done
    if [ $f -eq 0 ]
    then
        r=0
        num=$n
        while [ $num -gt 0 ] 
        do
        r=$((r*10+num%10))
        num=$((num/10))
        done
        if [ $r -eq $n ]
        then
            f=1
            echo "The next palindromic prime number is $n"
        else
            f=0
        fi
    else
        f=0
    fi
done

\end{Verbatim}
\newpage
\thispagestyle{third}
\subsection{Sample input and output}
\begin{figure}[H]
    \centering
    \includegraphics[width = 10cm ]{14.png}
    \caption{Output}
    \label{fig:my_label2}
\end{figure}

\section{\Large Result}
    Basic shell programs were written and executed successfully.

\end{document}