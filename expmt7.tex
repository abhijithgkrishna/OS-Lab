\documentclass{article}
\usepackage{float}
%\usepackage{hyperref}
\usepackage{fancyhdr}
\usepackage{graphicx}
\usepackage{fancyvrb}
%\usepackage[T1]{fontenc} 
\fancypagestyle{first}{
  \lhead{Experiment 7
  
  Date 13-03-2023}
  \rhead{}}
  \fancypagestyle{third}{
  \lhead{}
  \rhead{CPU Scheduling Algorithms}}
\setcounter{page}{35}
% Report TITLE
\title{\textbf {CPU Scheduling Algorithms}}
\date{\vspace{-5ex}}
\begin{document}
\maketitle
\thispagestyle{first}
\pagestyle{third}
%\pagestyle{first}
\section*{\Large Aim}
Implement the following CPU scheduling algorithms.\\
a) Round Robin\\
b) SJF\\
c) FCFS\\
d) Priority\\
Using these algorithm find turnaround time and waiting time. For all
the programs, arrival time for the processes should be considered and
a Gantt chart should be displayed as output. Output should be verified
with atleast 5 processes.\\
%\section{\Large fork}
\section{\Large Algorithm}
%\begin{Verbatim}[tabsize = 4]

%\end{Verbatim}
\subsection{Round Robin}
\begin{Verbatim}[tabsize = 4]
Step 1:Start
Step 2:Read the number of processes and their arrival time and burst time.
Step 3:Read the time quantum.
Step 4:Sort the processes based on their arrival time in increasing order.
Step 5:Initialize the current time to 0 and enqueue the first process in 
       the ready queue.
Step 6:While there are processes in the ready queue, do the following:
       a. Dequeue the next process from the ready queue.
       b. If the arrival time of the current process is greater than the current 
       time,set the current time to the arrival time of the current process.
       c. Execute the current process for a time quantum or until it finishes,
       whichever comes first. Subtract the execution time from the remaining time 
       of the process.
       d. Update the waiting time and turnaround time of the process based on 
       whether it has finished or not.
       e. Enqueue any newly arrived processes that can be executed in the 
       remaining time of the current process.
       f. If there are no more processes in the ready queue, set the current 
       time to the arrival time of the next process.
Step 7:Print the process table and the average waiting time and turnaround time.
Step 8:Stop

\end{Verbatim}
\subsection{Shortest Job First}
\begin{Verbatim}[tabsize = 4]
Step 1:Start
Step 2:Sort the processes based on their burst time in ascending order.
Step 3:Initialize the current time to 0 and the completion time to 0.
Step 4:While there are still processes in the ready queue, do the following:
       a.Select the process with the shortest burst time from the queue.
       b.Set its waiting time as the current time minus its arrival time.
       c.Update the completion time as the current time plus its burst time.
       d.Calculate its turnaround time as its completion time minus its arrival time.
       e.Remove the process from the ready queue.
       f.Repeat steps a to e until all processes have been completed.
Step 5:Calculate the average waiting time and average turnaround time for all processes.
Step 6:Print the results.
Step 7:Stop
\end{Verbatim}
\subsection{First come first serve}
\begin{Verbatim}[tabsize = 4]
Step 1 : Start
Step 2 : Read the number of processes (n) and their arrival time (at) and 
         burst time (bt).
Step 3 : Set the current time (ct) to 0 and initialize the waiting time (wt) 
         and turnaround time (tat) to 0 for all processes.
Step 4 : Sort the processes based on their arrival time in ascending order.
Step 5 : For each process in the sorted list, do the following:
         a.Update the current time to the arrival time of the process, if the 
         process arrives after the current time.
         b.Set the waiting time of the process to the current time.
         c.Update the current time by adding the burst time of the process.
         d.Set the turnaround time of the process to the difference between the 
         current time and the arrival time of the process.
Step 6 : Calculate the average waiting time and average turnaround time by summing 
         up the waiting time and turnaround time of all processes respectively and 
         dividing by the number of processes.
Step 7 : Output the waiting time, turnaround time, average waiting time, and average
         turnaround time.
Step 8 : Stop
\end{Verbatim}
\subsection{Priority Scheduling}
\begin{Verbatim}[tabsize = 4]
Step 1:Start
Step 2:Input the number of processes and their details including arrival time,
        burst time, and priority.
Step 3:Sort the processes in ascending order based on their arrival time.
Step 4:Set the current time (ct) to 0 and initialize an empty queue for the 
       ready processes.Loop until all processes have completed:
       a.Check if there are any processes that have arrived at the current time.
       If so, add them to the ready queue.
       b.If the ready queue is not empty, select the process with the highest 
       priority from the queue and execute it. Update the current time to the 
       completion time of the process.
       c.Calculate the waiting time and turnaround time for the executed process.
       d.Remove the executed process from the queue.
Step 5:Calculate the average waiting time and average turnaround time for all 
       processes.
Step 6:Stop
\end{Verbatim}

\section{Program}
\begin{Verbatim}[tabsize = 4]

#include<stdio.h>
#include<math.h>
#include<stdlib.h>

typedef struct process {
int pid, arrival, burst, wait, turn;
int temp, priority;
}
process;
int k = 0;
int * pid;
int * time;

void sort(process * p, int n) {
    for (int i = 0; i < n - 1; i++) {
int min = i;
        for (int j = i + 1; j < n; j++) {
            if (p[min].arrival > p[j].arrival)
                min = j;
        }
        if (i != min) {
            process temp = p[i];
            p[i] = p[min];
            p[min] = temp;
        }
    }
}
void fcfs(process * p, int n) {
    p[0].wait = 0;
    p[0].turn = p[0].burst;
    int sum = p[0].arrival;
    for (int i = 1; i < n; i++) {
        sum += p[i - 1].burst;
        p[i].wait = sum - p[i].arrival;
        if (p[i].wait < 0)
p[i].wait = 0;
        p[i].turn = p[i].burst + p[i].wait;
    }
}
 void gantt_chart(process * p, int n) {
    printf("\n\n");
    for (int i = 0; i < p[n - 1].turn - 1; i++) {
        printf(" ");
    }
    printf("GANTT CHART\n\n ");
    int i, j;
    for (i = 0; i < n; i++) {
        for (j = 0; j < p[i].burst; j++)
            printf("--");
        printf(" ");
    }
    printf("\n|");
    for (i = 0; i < n; i++) {
        for (j = 0; j < p[i].burst - 1; j++)
            printf(" ");
        printf("P%d", i + 1);
        for (j = 0; j < p[i].burst - 1; j++)
            printf(" ");
        printf("|");
    }
    printf("\n ");
    for (i = 0; i < n; i++) {
        for (j = 0; j < p[i].burst; j++)
            printf("--");
        printf(" ");
    }
    printf("\n%d", p[0].arrival);
    for (i = 0; i < n; i++) {
        for (j = 0; j < p[i].burst; j++)
            printf("  ");
        if (p[i].turn + p[i].arrival > 9)
            printf("\b");
        printf("%d", p[i].turn + p[i].arrival);
    }
    printf("\n");

}
void average(process * p, int n) {
    float sumw = 0, sumt = 0;
    for (int i = 0; i < n; i++) {
        sumw += p[i].wait;
        sumt += p[i].turn;
    }
    printf("\n\nAverage Waiting Time = %0.3f", sumw / n);
    printf("\nAverage Turnaround Time = %0.3f\n", sumt / n);
}
void fcfsmain() {
    int n;
    printf("Enter the number of processes: ");
    scanf("%d", & n);
    process p[10];
    printf("Enter the arrival times of Processes:");
    for (int i = 0; i < n; i++)
        scanf("%d", & p[i].arrival);
    printf("Enter the burst times of Processes:");
    for (int i = 0; i < n; i++) {
        scanf("%d", & p[i].burst);
        p[i].pid = i + 1;
    }
    sort(p, n);
    fcfs(p, n);
   printf("\n\n First Come First Served(FCFS) scheduling");
printf("\n-------------------------------------------------------------
----------------\n");
printf("Processes Arrival time\tBurst time\tWaiting time\tTurnaround time\n");
printf("-------------------------------------------------------------
----------------\n");
    for (int i = 0; i < n; i++) {
        printf(" P%d\t\t%d\t\t%d\t\t%d\t\t%d\n", p[i].pid, p[i].arrival, p[i].burst,
        p[i].wait, p[i].turn);
    }

    printf("-----------------------------------------------------------------
    ------------\n");
    gantt_chart(p, n);
    average(p, n);
}
void sort_arrival_time(process * p, int n) {
    for (int i = 0; i < n - 1; i++) {
        int min = i;
        for (int j = i + 1; j < n; j++) {
            if (p[min].arrival > p[j].arrival)
                min = j;
        }
        if (i != min) {
            process temp = p[i];
            p[i] = p[min];
            p[min] = temp;
        }
    }
}
void round_robin(process * p, int n, int quantum) {
    int rem[n];
    for (int i = 0; i < n; i++)
        rem[i] = p[i].burst;
    int previous = p[0].arrival;
    while (1) {
        int flag = 0;
        int i;
        int min = 0;
        for (i = 0; i < n; i++) {

            if (rem[i] > 0) {
                if (p[i].temp <= p[min].temp)
                    min = i;
            }
        }
        if (rem[min] > 0) {
            flag = 1;
            if (rem[min] > quantum) {
                previous += quantum;
                rem[min] -= quantum;
            } else {
                previous += rem[min];
                p[min].wait = previous - p[min].burst - p[min].arrival;
                rem[min] = 0;
                p[min].turn = p[min].burst + p[min].wait;
            }
            p[min].temp = previous;
            pid[k] = p[min].pid;
            time[k++] = p[min].temp;
        }
        if (!flag)
            break;
    }
}
void gantt(process * p, int n, int size) {
    printf("\n\n");
    for (int i = 0; i < (time[size - 1] * 2 + size - 11) / 2; i++)
        printf(" ");
    printf("GANTT CHART\n\n ");
    int i, j;
    for (i = 0; i < size; i++) {
        for (j = 0; j < time[i] - time[i - 1]; j++)
            printf("--");
        printf(" ");
    }
    printf("\n|");
    for (i = 0; i < size; i++) {
        for (j = 0; j < time[i] - time[i - 1] - 1; j++)
            printf(" ");
        printf("P%d", pid[i]);
        for (j = 0; j < time[i] - time[i - 1] - 1; j++)
            printf(" ");
        printf("|");
    }
    printf("\n ");
    for (i = 0; i < size; i++) {
        for (j = 0; j < time[i] - time[i - 1]; j++)
            printf("--");
        printf(" ");
    }
    printf("\n%d", p[0].arrival);
    for (i = 0; i < size; i++) {
        for (j = 0; j < time[i] - time[i - 1]; j++)
            printf("  ");
        if (time[i] > 9)
            printf("\b");
        printf("%d", time[i]);
    }
    printf("\n");
}

void rrmain() {
    process * p;
    int n, quantum, size = 0;
    printf("Enter the number of processes: ");
    scanf("%d", & n);
    p = (process * ) malloc(sizeof(process) * n);
    printf("Enter the time quantum : ");
    scanf("%d", & quantum);
    printf("Enter the arrival times of Processes:");

    for (int i = 0; i < n; i++)
        scanf("%d", & p[i].arrival);
    printf("Enter the burst times of Processes:");
    for (int i = 0; i < n; i++) {
        scanf("%d", & p[i].burst);
        p[i].pid = i + 1;
        p[i].temp = p[i].arrival;
        float f = p[i].burst / (float) quantum;
        size += ceil(f);
    }
    pid = (int * ) malloc(sizeof(int) * size);
    time = (int * ) malloc(sizeof(int) * size);
    sort_arrival_time(p, n);
    round_robin(p, n, quantum);
    printf("\n\n ROUND ROBIN SCHEDULING");
printf("\n---------------------------------------------------------
--------------------\n");
printf("Processes Arrival time\tBurst time\tWaiting time\tTurnaround time\n");
printf("-------------------------------------------------------------
----------------\n");
    for (int i = 0; i < n; i++) {
        printf(" P%d\t\t%d\t\t%d\t\t%d\t\t%d\n", p[i].pid, p[i].arrival, 
        p[i].burst, p[i].wait, p[i].turn);
    }
    printf("----------------------------------------------------------------
    -------------\n");
    gantt(p, n, size);
    average(p, n);
}
void priority(process * p, int n) {
    int previous = p[0].arrival;
    for (int i = 0; i < n; i++) {
        int min = p[i].priority;
        int min_index = i;
        int flag = 0;
        for (int j = i + 1; j < n; j++) {
            if (p[j].arrival > previous)
                break;
            if (p[j].priority < min) {
                min = p[j].priority;
                min_index = j;
                flag = 1;
            }

        }
        if (flag == 1) {
            process temp = p[i];
            p[i] = p[min_index];
            p[min_index] = temp;
        }
        p[i].wait = previous - p[i].arrival;
        if (p[i].wait < 0)
            p[i].wait = 0;
        p[i].turn = p[i].burst + p[i].wait;
        previous = p[i].arrival + p[i].turn;
    }
}

void pmain() {
    int n;
    printf("Enter the number of processes: ");
    scanf("%d", & n);
    process p[10];
    printf("Enter the arrival times of Processes:");
    for (int i = 0; i < n; i++)
        scanf("%d", & p[i].arrival);
    printf("Enter the burst times of Processes:");
    for (int i = 0; i < n; i++) {
        scanf("%d", & p[i].burst);
        p[i].pid = i + 1;
    }
    printf("Enter the priorities of Processes:");
    for (int i = 0; i < n; i++) {
        scanf("%d", & p[i].priority);
    }

    sort_arrival_time(p, n);
    priority(p, n);
    printf("\n\n PRIORITY SCHEDULING");
printf("\n-------------------------------------------------------------------
---------------------\n");
printf("Processes Arrival time\tBurst time\tPriority\tWaiting time\tTurnaround time\n");
printf("-----------------------------------------------------------------------
-----------------\n");
    for (int i = 0; i < n; i++) {
        printf(" P%d\t\t%d\t\t%d\t %d\t\t%d\t\t%d\n",
            p[i].pid, p[i].arrival, p[i].burst, p[i].priority,
            p[i].wait, p[i].turn);
    }
    printf("----------------------------------------------------------------
    -------------------\n");
    gantt_chart(p, n);
    average(p, n);
}

void sjfs(process * p, int n) {
    int previous = p[0].arrival;
    for (int i = 0; i < n; i++) {
        int min = p[i].burst;
        int min_index = i;
        int flag = 0;
        for (int j = i + 1; j < n; j++) {
            if (p[j].arrival > previous)
                break;
            if (p[j].burst < min) {
                min = p[j].burst;
                min_index = j;
                flag = 1;

            }
        }
        if (flag == 1) {
            process temp = p[i];
            p[i] = p[min_index];
            p[min_index] = temp;
        }
        p[i].wait = previous - p[i].arrival;
        if (p[i].wait < 0)
            p[i].wait = 0;
        p[i].turn = p[i].burst + p[i].wait;
        previous = p[i].arrival + p[i].turn;
    }
}

void sjfsmain() {
    int n;
    printf("Enter the number of processes: ");
    scanf("%d", & n);
    process p[10];
    printf("Enter the arrival times of Processes:");
    for (int i = 0; i < n; i++)
        scanf("%d", & p[i].arrival);
    printf("Enter the burst times of Processes:");
    for (int i = 0; i < n; i++) {
        scanf("%d", & p[i].burst);
        p[i].pid = i + 1;
    }
    sort_arrival_time(p, n);
    sjfs(p, n);
    printf("\n\n Shortest Job First(SJF) SCHEDULING");
printf("\n-------------------------------------------------------
----------------------\n");
printf("Processes Arrival time\tBurst time\tWaiting time\tTurnaround time\n");
printf("--------------------------------------------------------------
---------------\n");
    for (int i = 0; i < n; i++) {
        printf(" P%d\t\t%d\t\t%d\t\t%d\t\t%d\n", p[i].pid,
            p[i].arrival, p[i].burst, p[i].wait, p[i].turn);
    }
    printf("--------------------------------------------------------
    ---------------------\n");
    gantt_chart(p, n);
    average(p, n);
}

void main() {
    printf("Choose algorithm : \n");
    int choice;
    printf("1.Round Robin\n2.FCFS \n3.SJF \n4.Priority SCHEDULING\n5.exit\n");
    printf("Enter choice : ");
    scanf("%d", &choice);
    while(choice!=5)
     {
    if (choice == 1)
        rrmain();
    else if (choice == 2)
        fcfsmain();
    else if (choice == 3)
        sjfsmain();
    else if (choice == 4)
        pmain();
    printf("Enter choice: ");
    scanf("%d",&choice);
     }
}
\end{Verbatim}

\section{Sample Input and Output}
\begin{figure}[H]
    \centering
    \includegraphics[width = 10cm ]{3/s.1.png}
    \caption{Output}
    \label{fig:my_label2}
\end{figure}
\begin{figure}[H]
    \centering
    \includegraphics[width = 10cm ]{3/s.2.png}
    \caption{Output}
    \label{fig:my_label2}
\end{figure}
\begin{figure}[H]
    \centering
    \includegraphics[width = 10cm ]{3/s.3.png}
    \caption{Output}
    \label{fig:my_label2}
\end{figure}
\begin{figure}[H]
    \centering
    \includegraphics[width = 10cm ]{3/s.4.png}
    \caption{Output}
    \label{fig:my_label2}
\end{figure}

\section{\Large Result}
    Excecuted cpu scheduling algorithms

\end{document}
